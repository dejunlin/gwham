\section{Usage} \label{sec:usage}
The user can use the WHAM kernel as a library in his/her own
program or to use the ``gwham'' program to compute the thermodynamic
quantity of interest. 

\subsection{Use the WHAM kernel}
The user is strongly encouraged to read section~\ref{sec:impl} before using the
WHAM kernel. The WHAM kernel is implemented in the C++ template class ``WHAM''
in ``gwham.hpp''. All the WHAM calculation is done upon instantiation of this
class, i.e., all the function calls are directly inside or nested in the
constructor of the ``WHAM'' class. The ``WHAM'' class takes three template
arguments: 1) the pointer to the generic ensemble class; 2) histogram class; 3)
multidimensional array class. Its constructor expects a list of arguments
described in the comments right over its declaration in ``gwham.hpp'' and will
not be elaborated here. The description is straightforward and the user should
read it carefully. Since the ``WHAM'' class is a template, there's no need to 
link against it upon the compilation of whatever program the user is writing. 
However, the user needs to link his/her program against the ``Hamiltonian'' 
and ``Ensemble'' class objects (compiled from the respective ``hamiltonian.cpp''
and ``ensemble.cpp'' file) if he/she uses it.

\subsection{Use the ``gwham'' executable}
The ``gwham'' executable reads a set of simulation parameter files, parse the
time series along the reaction coordinates (RCs) and compute the potential of
mean force (PMF) along the RCs. For now, it can only read parameter and 
time series files in GROMACS format but the users can always transform their 
data into the GROMACS format in order to use it (see below).

\subsubsection{File naming convention}\label{sec:fname}
Each simulation parameter file corresponds to one simulation window and tells
the ``gwham'' program how to read the time series, what the RCs and the
Hamiltonians are. The time series are the data output from the simulation
software and usually are the RC values as a function of time. The simulation
parameter files and the time series files must be in ASCII and must be named by
the following convention: ``prefix\_window\_run[suffix]'', where ``prefix'' is
any arbitrary string that identifies this set of simulations and is provided as
a command line argument to the ``gwham'' program (see below) and ``window'' is
an index that is consistent between a simulation parameter file and a
consecutive set of time series file which are further indexed by ``run''. The
index ``run'' is there for the users' convenience because usually a simulation
is run in chunks and produces a consecutive set of output files; ``run'' thus
tells the program to identify each chunk by this unique index. Both ``window''
and ``run'' are non-negative integers that start from 0. ``[suffix]'' is a
string that's dependent on the specific MDP class (see section~\ref{sec:api})
and tells the program what file it's about to read.  For example, if we have a
set of $\mathbf{10}$ umbrella sampling simulations (``called foo'') in GROMACS
with each consists of $\mathbf{5}$ runs, the simulation parameter files should
be named as: ``foo\_0.mdp'', ``foo\_1.mdp'', $\cdots$, ``foo\_9.mdp'' and for
the $i$'th ``mdp'' file, there should be $M$ time series files:
``foo\_i\_0x.xvg'', ``foo\_i\_0x.xvg'', $\cdots$, ``foo\_i\_4x.xvg'' (where the
``x.xvg'' suffix is expected by the ``GMXMDP'' class). 

\subsubsection{Supported GROMACS free energy or PMF calculation schemes}
The following GROMACS free energy calculation schemes are currently supported, i.e., 
the options in ``mdp'' files that can be understood by the ``gwham'' program:

\begin{enumerate}
  \item \label{it:umb} PMF calculation via umbrella sampling 
  \item \label{it:hrex} PMF calculation via umbrella sampling with Hamiltonian replica exchange 
  \item \label{it:expens} PMF calculation via umbrella sampling in expanded ensemble 
\end{enumerate}

\change{Note that features~\ref{it:hrex} and \ref{it:expens} are only available
for a in-house version of GROMACS 4.6.3 developed by me. Also note although
``gwham'' can handle cases where the temperatures and/or pressures in different
simulation windows vary, e.g., in a temperature replica exchange simulation or
in a umbrella sampling simulation coupled to temperature replica exchange, the
program further expects a set of ``ener.xvg'' files in addtion to the ``x.xvg''
files and I have not tested such cases yet and the users might need to contact
me for further instruction on this.} 

\subsubsection{How to prepare input files for ``gwham''}\label{sec:fformat}
For users who want to use ``gwham'' to analyse the data from GROMACS, it's
sufficient to rename their files following the convention in
section~\ref{sec:fname}. Alternatively,  you can generate symbolic links to the
data according to the naming convention so you can keep your original data
without renaming them. The syntax of the simulation parameter files (``mdp''
files) and the time series ``x.xvg'' files should following strictly those
documented in the GROMACS 4.6 manual. 

Alternative to the time series file, the user can also supply the histograms
directly so that the program won't need to transform the time series into
histograms. There should be one histogram for each state/ensemble. Note that
the number of state doesn't necessarily equal the number of simulations, e.g.,
in a expanded ensemble simulation, and is automatically determined by the
program when analysing the ``mdp'' files.  The histogram file should contain
$2N+1$ columns, where $N$ is the dimensionality of the RCs, with the first $N$
columns denotes the indices of the multidimensional grid, the following $N$
columns denotes the RC values of the corresponding grid point and the last
column denotes the number of samples in that grid point. Only grid points with 
at least 1 sample needs to be specified.

For those who are not GROMACS users, you need transform your original data into
GROMACS 4.6 format in order to use ``gwham''. If you're interested, I strongly
recommend you to read this documentation as well as the GROMACS 4.6 manual
carefully and you're welcomed to contact me if you need any help. 

\subsubsection{Command line arguments}
It expects the list of arguments in the following order:

\begin{enumerate}
  \item File name prefix for simulation parameter files and time series file
  (see section~\ref{sec:fname} for what this means)
  \item The simulation parameter file suffix that tells ``gwham'' what types of
  simulation package you're using -- \change{for now this has to be the string
  ``mdp'', i.e., only GROMACS ``mdp'' files can be parsed.}
  \item Number of simulation windows (see section~\ref{sec:fname} for what this
  means)
  \item Number of runs for each window (see section~\ref{sec:fname} for what
  this means). Note that you can simply put a large enough integer here so that
  the program will read as many runs as possible.
  \item The indices (starting from $0$) of the RCs that you want to calculate
  PMFs on. E.g., if you performed a 5-dimensional umbrella sampling and you
  want to calculate the PMF along the 1st, the 3rd and the 4th dimension, you
  should use ``0 2 3'' here.
  \item Number of bins (deliminated by spaces) to use for histogramming for
  each of the dimension of the RCs. E.g., if your umbrella sampling simulation
  is along a 3-dimensional RC, you should use ``$n_{1} n_{2} n_{3}$'' to
  specify how many bins you want.
  \item Upper bounds (deliminated by spaces) to use for histogramming for
  each of the dimension of the RCs.
  \item lower bounds (deliminated by spaces) to use for histogramming for
  each of the dimension of the RCs.
  \item Tolerance for the error in converging the solution to WHAM equation.
  \item Which data point onwards to be regarded as the first sample in all the
  time series file.  If you want to exclude the first $N$ data points in all
  the time series, you should put $N$ here. Note that the comments are not
  counted as valid data since they are automatically ignored by the parser.
  \item How frequently we skip reading data points in the time series, i.e.,
  only read the time series only every this number of data points.
  \item Which data point to be the last one included in the analyse. All the 
  samples after this are excluded.
  \item The name of a ``mdp'' file (without the ``.mdp'' suffix) that specifies 
  the thermodynamic parameters of a ensemble where we want the PMF to be calculated.
  E.g., if you want the PMF to be in $310$~K instead of $300$~K, you should prepare
  a ``mdp'' file with the temperature specified as $310$~K.
  \item A file contains the seeding values for the free energy in each state/ensemble.
  The free energy values must be specified in one line deliminated by spaces. 
  The number of values should match the number of states in the simulations or otherwise
  the content of this file will be ignored. Note that the number of state doesn't 
  necessarily equal the number of simulations, e.g., in a expanded ensemble simulation, 
  and is automatically determined by the program when analysing the ``mdp'' files.
  \item A boolean value ($1$ or $0$) that tells the program if to perform
  minimization before WHAM iteration.
  \item A boolean value ($1$ or $0$) that tells the program if we can directly reads 
  in histogram instead of time series data. The format histogram file is documented in 
  section~\ref{sec:fformat}.
\end{enumerate}

\subsection{Output from the ``gwham'' executable}
The followings are printed to standard output when running ``gwham'':
\begin{enumerate}
  \item What type of command line arguments it's expecting
  \item The arguments that are given
  \item The names of the ``mdp'' file
  \item The simulation parameters that it successfully reads in
  \item The names of the ``x.xvg'' files (or potentially any time series files)
  or histogram files
  \item The histograms that it creates or reads
  \item The index of the histograms that contribute to each bin
  \item The overlap matrix (in percentage) between any pair of histograms
  \item Seeding free energies in each state
  \item Seeding free energy difference between any pair of states
  \item The free energy difference during optimization (printed out every 100
  steps of minimization)
  \item Indication of whether convergence has been met or not for minimization
  \item Free energies of each state during WHAM iteration (printed out every
  100 steps of iteration)
  \item Indication of whether convergence has been met for WHAM iteration
  \item PMF along the specified dimensions of RCs as well as the corresponding
  relative probability.  This contains $2N+2$ columns with the first $2N$ being
  the specification of the indices as well as the RC values for each grid point
  and the last two columns are the respective PMF and probability.
\end{enumerate}
