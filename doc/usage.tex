\section{Usage}
The user can use the WHAM kernel as a library in his/her own
program or to use the ``gwham'' program to compute the thermodynamic
quantity of interest. 

\subsection{Use the WHAM kernel}
The user is strongly encouraged to read section~\ref{sec:impl} before using the
WHAM kernel. The WHAM kernel is implemented in the C++ template class ``WHAM''
in ``gwham.hpp''. All the WHAM calculation is done upon instantiation of this
class, i.e., all the function calls are directly inside or nested in the
constructor of the ``WHAM'' class. The ``WHAM'' class takes three template
arguments: 1) the pointer to the generic ensemble class; 2) histogram class; 3)
multidimensional array class. Its constructor expects a list of arguments
described in the comments right over its declaration in ``gwham.hpp'' and will
not be elaborated here. The description is straightforward and the user should
read it carefully. Since the ``WHAM'' class is a template, there's no need to 
link against it upon the compilation of whatever program the user is writing. 
However, the user needs to link his/her program against the ``Hamiltonian'' 
and ``Ensemble'' class objects (compiled from the respective ``hamiltonian.cpp''
and ``ensemble.cpp'' file) if he/she uses it.
