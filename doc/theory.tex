\section{Theory} 
In order to combine the information from multiple systems, we resort to 
quantities which are conserved across all the systems. One obvious quantity 
is $\vect{q}$ as defined in equation~\ref{eq:q}. If we know $p_{i}(\vect{q})$, 
the probability density of $\vect{q}$, we can reformulate equations~\ref{eq:Q} 
and \ref{eq:O} in terms of $p_{i}(\vect{q})$:
\begin{equation}
Q_{i} = \intgrln{}{}{p_{i}}{\vect{q}}
\label{eq:Q_q}
\end{equation}
\begin{equation}
\eavg{O}_{i} = \frac{\intgrln{}{}{p_{i}(\vect{q})O}{\vect{q}}}{Q_{i}} 
\label{eq:Oavg_q}
\end{equation}
and $p_{i}(\vect{q})$ is defined as
\begin{equation}
p_{i}(\vect{q}) \equiv \Omega(\vect{q}) \exp(-H_{i}(\vect{q})) 
\label{eq:p_q}
\end{equation}
where
\begin{equation}
\Omega(\vect{q}) \equiv \intgrlnnoarg{}{}{\delta(\vect{q}-\vect{q}(\vect{x}))}{x}
\label{eq:omega}
\end{equation}
is the density of state at $\vect{q}$. $O(\vect{q})$ can be obtained by
\begin{equation}
O(\vect{q}) \equiv \frac{\intgrln{}{}{\delta(\vect{q}-\vect{q}(\vect{x}))O}{\vect{x}}}{Q_{i}}
\label{eq:O_q}
\end{equation}

WHAM works in the following way. First, we discretize the space of $\vect{q}$ 
into a $d \equiv \mathrm{dim}(\vect{q})$ -dimensional grid, where each grid point can be 
identified by a unique vector $\mathbf{m} \in \nspace{Z}{d}$. Now we denote the 
value of a function of $\mathbf{q}$: $f(\mathbf{q})$ takes in bin $\mathbf{m}$ 
as $f_\mathbf{m}$. If we have a total of $N_i$ samples from the $i$'th 
simulation where $n_{i,\mathbf{m}}$ are histogrammed into the bin $\mathbf{m}$. 
Then we combine the histograms from all $S$ simulations to optimally estimate 
$\Omega_{\mathbf{m}}$ and thus $p_{i,m}$. There're several approaches to arrive 
at the same set of WHAM equations that gives the optimal estimate of $\Omega_{\mathbf{m}}$: 
the original WHAM paper\cite{Kumar1992} as well as the PT/STWHAM one\cite{Chodera2007} 
treats $\Omega_\mathbf{m}$ as a weighted sum of $\Omega_{i,\mathbf{m}}$, 
the density of state in bin $\mathbf{m}$ estimated from the $i$'th simulation alone, 
where the weights are $\dev^2 \Omega_{i,\mathbf{m}}$, the respective statistical 
uncertainty of $\Omega_{i,\mathbf{m}}$ from each simulation. This approach introduces
the statistical uncertainty in the samples and help the development of basis error 
analysis. But the number of WHAM equations resulting from this approach is the 
same as number of bins, which takes a tremendous amount of computational resources 
to solve. Here we take the maximum likelihood approach instead, which was reported 
by Zhu and Hummer~\cite{Zhu2012}, because it shows a way to solve the same problem 
in terms of $Q_i$ in lieu of $\Omega_\mathbf{m}$, which dramatically reduces the 
number of parameters to determine and boost the computation significantly~\cite{Zhu2012}.
Although in that paper, their results only applied to calculating PMFs, which is
not an limitation on the discussion here.

The maximum likelihood approach maximize the following likelihood function given 
$\{p_{i,\mathbf{m}}\}_i$ and $\{n_{i,\mathbf{m}}\}_i$, where $\{f_i\}_i$ denotes the set 
of $f_i$ for all $i$:
\begin{equation}
A(\{n_{i,\mathbf{m}}\}_{i,\mathbf{m}} | \{p_{i,\mathbf{m}}\}_{i,\mathbf{m}}) \equiv \\
\prod_{i=1}^{S} P_{i}(\{n_{i,\mathbf{m}}\}_{\mathbf{m}} | \{p_{i,\mathbf{m}}\}_{\mathbf{m}})
\label{eq:A}
\end{equation}
where $A_{i}(\{n_{i,\mathbf{m}}\}_{\mathbf{m}} | \{p_{i,\mathbf{m}}\}_{\mathbf{m}})$ 
is the probability of observing the $i$'th histograms $\{n_{i,\mathbf{m}}\}_{\mathbf{m}}$ 
given $\{n_{i,\mathbf{m}}\}_{\mathbf{m}}$:
\begin{equation}
A_{i}(\{n_{i,\mathbf{m}}\}_{\mathbf{m}} | \{p_{i,\mathbf{m}}\}_{\mathbf{m}}) \equiv \\
\frac{(N_i)!}{\prod\limits_{\mathbf{m}}^{} (n_{i,\mathbf{m}})!} \\
\prod\limits_{\mathbf{m}}^{} (p_{i,\mathbf{m}}/Q_{i})^{n_{i,\mathbf{m}}}
\label{eq:Ai}
\end{equation}
We define our log-likelihood function $F$ from the relation:
\begin{equation}
\ln(A(\{\Omega_{i,\mathbf{m}}\}_{i,\mathbf{m}})) = - F(\{\Omega_{\mathbf{m}}\}_{\mathbf{m}}) + const.
\label{eq:F}
\end{equation}
where we treat all the quantities known during the analysis, such as 
$\{n_{i,\mathbf{m}}\}_{i,\mathbf{m}}$ and $\exp(-H_{i,\mathbf{m}})$, as constants.
Our optimal estimate of $\{\Omega_{i,\mathbf{m}}\}_{i,\mathbf{m}}$ would be the one 
that minimizes $F$. Substituting equations~\ref{eq:A} and \ref{eq:Ai} into \ref{eq:F} 
gives 
\begin{equation}
F(\{\Omega_{\mathbf{m}}\}_{\mathbf{m}}) = -\sum_{\mathbf{m}}^{} C_{\mathbf{m}} \ln \Omega_{\mathbf{m}} \\
- \sum_{i}^{} N_i \ln f_{i}
\label{eq:Fexw}
\end{equation}
where $C_{\mathbf{m}} \equiv \sum\limits_{i}^{} n_{i,\mathbf{m}}$ is the number of counts
in bin $\mathbf{m}$ and $f_{i}$ is inverse of $Q_{i}$ which takes the discretized form
\begin{equation}
f_{i} \equiv \frac{1}{\sum_{\mathbf{m}}^{} \Omega_{\mathbf{m}} \exp(-H_{i,\mathbf{m}}) M}
\end{equation}
and $M$ is the $d$-dimensional volume of each bin, which we assume to be constant across 
different $\mathbf{m}$. If we take the derivative of $F$ in equation~\ref{eq:Fexw} with
respect to $\Omega_{\mathbf{m}}$, we have
\begin{equation}
\frac{\partial F}{\partial \Omega_{\mathbf{m}}} = -C_{m}/\Omega_{m} + \sum_{i}^{} N_{i} f_{i} \exp(-H_{i,\mathbf{m}}) M
\label{eq:dFdw}
\end{equation}
By demanding equation~\ref{eq:dFdw} to zero we obtain the WHAM equations in the same form 
as those obtained before~\cite{Zhu2012,Chodera2007,Kumar1992}:
\begin{equation}
\Omega_{m} = \frac{C_{\mathbf{m}}}{\sum\limits_{i}^{} N_{i} f_{i} exp(-H_{i,\mathbf{m}}) M}
\label{eq:WHAM1}
\end{equation}
Since $\{f_{i}\}_{i}$ are functions of $\{\Omega_{\mathbf{m}}\}_{\mathbf{m}}$, equation~\ref{eq:WHAM1}
is a set of nonlinear equations which are usually solved by iteratively updating 
$\{f_{i}\}_{i}$ and $\{\Omega_{\mathbf{m}}\}_{\mathbf{m}}$ until self-consistency is 
achieved~\cite{Kumar1992,wham205}. We can also show that $F$ is a convex function that 
has one and only one global minimum.

To proceed, we re-interpret $F$ as a new function of $\{f_{i}\}_{i}$:
\begin{equation}
F(\{f_{i}\}_{i}) = - \sum_{i}^{} N_i \ln f_{i} \\ 
-\sum_{\mathbf{m}}^{} C_{\mathbf{m}} \ln \frac{C_{\mathbf{m}}}{z_{\mathbf{m}}} 
\label{eq:Fexf}
\end{equation}
or a function of $g_{i} \equiv \ln f_{i}$
\begin{equation}
F(\{g_{i}\}_{i}) = - \sum_{i}^{} N_i g_{i} \\ 
-\sum_{\mathbf{m}}^{} C_{\mathbf{m}} \ln \frac{C_{\mathbf{m}}}{z_{\mathbf{m}}} 
\label{eq:Fexg}
\end{equation}
with $c_{i,\mathbf{m}} \equiv M \exp(-H_{i,\mathbf{m}})$ and 
$z_{\mathbf{m}} \equiv z_{\mathbf{m}}(\{g_{i}\}_{i}) \equiv \sum\limits_{i}^{} N_{i} c_{i,\mathbf{m}} \exp(g_{i})$ 
for convenience. It's straightforward to show that
\begin{equation}
\frac{\partial F}{\partial g_{i}} = 0 \: \forall i \Leftrightarrow \frac{\partial F}{\partial \Omega_{\mathbf{m}}} \: \forall \mathbf{m}
\label{eq:Fequiv}
\end{equation}
To show that $F(\{g_{i}\}_{i})$ is a convex function of $\{g_{i}\}_{i}$ and thus has one 
global minimum, we first give the first and second partial derivatives of $F$:
\begin{equation}
\frac{\partial F}{\partial g_{i}}  = N_{i} (\exp(g_{i}) \sum_{\mathbf{m}}^{} \frac{C_{\mathbf{m}} c_{i,\mathbf{m}} }{z_{\mathbf{m}}} -1)
\label{eq:dFdg}
\end{equation}
\begin{equation}
\frac{\partial^2 F }{\partial g_{i} \partial g_{j}} =  \\
\delta_{ij} N_{i} \exp(g_{i}) \sum_{\mathbf{m}}^{} \frac{C_{\mathbf{m}} c_{i,\mathbf{m}}}{z_{\mathbf{m}}} - \\
N_{i} N_{j} \exp(g_{i}+g_{j}) \sum_{\mathbf{m}}^{} \frac{C_{m}c_{i,\mathbf{m}}c_{j,\mathbf{m}}}{z_{\mathbf{m}}^2}
\label{eq:dFdg2}
\end{equation}
where $\delta_{ij}$ is the kronecker delta. Let $\mathbf{B}$ be the Hessian matrix of $F$. We 
have $\mathbf{B} = \mathbf{C} - \sum\limits_{\mathbf{m}}^{} C_{\mathbf{m}} \mathbf{D}_{\mathbf{m}} \otimes \mathbf{D}_{\mathbf{m}}$ 
where $\mathbf{C}$ is a diagonal matrix with the main diagonal being 
$\{N_{i} \exp(g_{i}) \sum\limits_{\mathbf{m}}^{} \frac{C_{\mathbf{m}} c_{i,\mathbf{m}}}{z_{\mathbf{m}}}\}_{i}$, 
the first term in equation~\ref{eq:dFdg2}, and $\mathbf{D}_{\mathbf{m}} \equiv \{N_{i} \exp(g_{i}) c_{i,\mathbf{m}}/z_{\mathbf{m}}\}_{i}$.
Next we'll show $\mathbf{a}^{\mathrm{T}} \mathbf{B} \mathbf{a} \ge 0\;\forall \mathbf{a} \in \nspace{R}{K}$.
\begin{align}
\mathbf{a}^{\mathrm{T}} \mathbf{B} \mathbf{a} &\ge 0 \Leftrightarrow \notag \\
\mathbf{a}^{\mathrm{T}} \mathbf{C} \mathbf{a} &\ge \sum_{\mathbf{m}}^{} C_{\mathbf{m}}  \mathbf{a}^{\mathrm{T}} \mathbf{D}_{\mathbf{m}} \otimes \mathbf{D}_{\mathbf{m}} \mathbf{a}  \Leftrightarrow \notag \\
\mathbf{a}^{\mathrm{T}} \mathbf{C} \mathbf{a} &\ge \sum_{\mathbf{m}}^{} C_{\mathbf{m}} (\mathbf{a} \cdot \mathbf{D}_{\mathbf{m}})^2 \Leftrightarrow \notag \\
\sum_{\mathbf{m}}^{} \frac{\sum\limits_{i}^{} a_{i}^{2} N_{i} \exp(g_{i}) C_{\mathbf{m}} c_{i,\mathbf{m}}}{z_{\mathbf{m}}} &\ge  
\sum_{\mathbf{m}}^{} \frac{C_{\mathbf{m}}(\sum_{i}^{} N_{i} c_{i,\mathbf{m}} \exp(g_{i}) a_{i})^{2}}{z_{\mathbf{m}}^{2}} \Leftarrow \notag \\
z_{\mathbf{m}} \sum\limits_{i}^{} a_{i}^{2} N_{i} \exp(g_{i}) c_{i,\mathbf{m}} &\ge  
(\sum_{i}^{} N_{i} c_{i,\mathbf{m}} \exp(g_{i}) a_{i})^{2}, \;\forall \mathbf{m} \Leftrightarrow \notag \\
\vnorm{\mathbf{s}_{\mathbf{m}}}^{2} \vnorm{\mathbf{t}_{\mathbf{m}}}^{2} &\ge \vnorm{\mathbf{s}_{\mathbf{m}} \cdot \mathbf{t}_{\mathbf{m}}}^{2}  
\label{eq:derv1}
\end{align}
where
\begin{align}
\mathbf{s}_{\mathbf{m}} &\equiv \{\sqrt{N_{i} c_{i,\mathbf{m}} \exp(g_{i})}\}_{i}, \notag \\
\mathbf{t}_{\mathbf{m}} &\equiv \{a_{i} \sqrt{N_{i} c_{i,\mathbf{m}} \exp(g_{i})}\}_{i}
\label{eq:derv2}
\end{align}
and the last step in equation~\ref{eq:derv1} follows the Cauchy-Schwarz inequality.  
The twice differentiable function $F$ has positive semidefinite Hessian for all $\{g_{i}\}_{i}$ 
and thus is convex.
