\section{Implementation}\label{sec:impl}
\change{The purpose of this section is not to elaborate the details of my
implementation, as I don't and will not have the intention/time to do so,
but to provide a general understanding of how my implementation works or how
one could potentially implement their own WHAM program}. The current WHAM
program I wrote contains a series of application program interfaces (APIs) and
a kernel. The APIs prepare and present the data to the kernel while the kernel
is responsible of solving equation~\ref{eq:WHAM1}.

\subsection{Kernel}
There are three major steps in the kernel: 
1) Caching $\{N_{i}\exp(-H_{i,\mathbf{m}})\}_{i}$ in memory;
2) Minimize function $F$ in equation~\ref{eq:Fexg}. This step is an optional;
3) Solve for equation~\ref{eq:WHAM1} iteratively. (Note that the iteration is
virtually skipped if step one results in a converged answer.) 
I will give more details about these steps as well as the user interface
in my implementation in the following sections. 

\subsubsection{Caching $\{N_{i}\exp(-H_{i,\mathbf{m}})\}_{i}$}
Because evaluating the exponential function is in general an expensive 
operation, we can precompute $\{\exp(-H_{i,\mathbf{m}})\}_{i}$ or more effectively, 
$\{N_{i}\exp(-H_{i,\mathbf{m}})\}_{i}$ before we do any minimization of iteration
to save computation time. One caveat is that doing this could be 
memory demanding especially when we're dealing with high-dimensional 
reaction coordinate space with a large number of bins. My recommendation 
is to perform initial WHAM runs with a small number of bins to obtain 
a set of $\{g_{i}\}_{i}$, which can then be used to seed another round
of WHAM with larger number of bins. My implementation will cache
$\{N_{i}\exp(-H_{i,\mathbf{m}})\}_{i}$ in any case and it's thus worth watching 
for the memory requirement in the initial run.

\subsubsection{Minimizing $F$ in equation~\ref{eq:Fexg}}
As shown by Zhu and Hummer~\cite{Zhu2012}, it's more efficient to minimize $F$
as a function of $\Delta g_{ij} \equiv g_{i} - g_{j}$.  The form of such
function was documented in their paper~\cite{Zhu2012} except that we need to
replace the harmonic potential with the generalized Hamiltonian in
section~\ref{sec:theory}. However, since we are dealing with a generalized form
of Hamiltonian and typically multidimensional reaction coordinate, simply
taking $i$ and $j$ as the nearest neighbors of sampling windows won't lead to
high enough computational efficiency to justify the extra cost of calculating
the gradient of $F$ as compare to simply solving equation~\ref{eq:WHAM1}
iteratively. Although I've not come up with an analytic justification for
pairing of $i$ and $j$ in general, I often found in practice that pairing $i$
and $j$ when the corresponding histograms maximally overlap with each other
works very well. This can be seen from equation~\ref{eq:dFdg} where
$\frac{\partial F}{\partial g_{i}}$ is predominantly determined by those $j$
which have significant number of samples in the bins $\mathbf{m}$ where
$c_{i,\mathbf{m}}$ is non trivial, i.e., $i$ and $j$ histograms overlap significantly.
Thus, the first step to minimization of $F$ is to construct a nearest neighbor 
list of $g_{i}$ and $g_{j}$ where the corresponding histograms overlap maximally.
This results in a tree-like structure with each $g_{i}$ being the node and 
each $\Delta g_{ij}$ being the edge linking the two nodes $g_{i}$ and $g_{j}$.
Note that inside my implementation, such a tree is not implemented in the 
usual sense of tree structure in most computer program; it's rather a 
``aggregation'' of nodes and edges that recognize one another by a map. 
The reason I did this is I couldn't find any reliable C++ tree structure 
implementation that meets my need. However, the construction of such a 
tree needs to be done only once and is rather cheap operation in most of 
the application to date (considering that most of the sampling simulations
have less than $1000$ sampling windows to be analyzed by WHAM). After
we got all the $\Delta g_{ij}$, the calculation of $F$ and its gradient
is straightforward. It's thus easy to use any of of the minimization 
algorithm to minimize $F$. In my implementation, I used Polak-Ribiere 
conjugate gradient method with Brent’s line search~\cite{NumRec2007} but 
the adaptation to other methods should be straightforward. 

\subsubsection{Solve equation~\ref{eq:WHAM1}}
The procedure of solving equation~\ref{eq:WHAM1} is very straightforward and
will not be elaborated here. I just want to mention that in my implementation
the evaluation of $\Omega_{\mathbf{m}}$ is in its own module module called the
``DensityOfState'' so that the ``WHAM'' class is only responsible for checking
the self-consistency in equation~\ref{eq:WHAM1}.  The ``DensityOfState'' class
also provides some basic application program interfaces (APIs) such as the
square bracket operator for computing other thermodynamic quantities as
mentioned in section~\ref{sec:intro}.

\subsection{Application Program Interfaces (APIs)}\label{sec:api}
The section describes the APIs for preparing necessary data for WHAM. The three
major components of APIs are parsing the sampling data, creating the histogram 
and setting the Hamiltonian.

The sampling data is usually a time series generated from the simulation
program. In the current implementation, the time series are not needed by WHAM
kernel \textit{per se} but are directly binned into histogram. The data parser 
is instantiated by a generic interfacial class (see below). 

Histograms are represented by a multidimensional array implemented in the
``gnarray'' class. The dimensionality is determined at run time (by the output 
from the time-series data parser). The underlying data structure of the 
``gnarray'' class is a map from the reaction coordinate values (or the bin)
to a integer, i.e., the number of counts in the bin. Basic accessor APIs
to the underlying array element are provided.

The Hamiltonian is basically a functor that maps the reaction coordinates 
to the potential energy. The class ``Hamiltonian'' provides APIs to 
compute the potential energy for all the reaction coordinate bins. It's 
generalized to handle a wide range of potential energy functions. The 
Hamiltonian is wrapped inside another class called ``Ensemble'', 
which provides generic interface to access the Hamiltonian of different 
thermodynamic ensembles. The ``Ensemble'' class along with its member
``Hamiltonian'' is instantiated by a generic interfacial class (see below).

To better organize the codes and provide more user-friendly interface, I also
provide a generic interfacial class to link the 3 aforementioned components
together. This interfacial class is called ``MDP'' (meaning Molecular Dynamics
Parameter). It reads the software specific parameter file, create and instruct
a time-series file reader and construct the necessary Hamiltonian. This way, a
generic main function can be used to handle different output files from
different simulation software. The main function's job is just to check for the
consistency of command line arguments, to instantiate the ``MDP'' class and
pass the parameter files to it, to use the time-series file reader from the
``MDP'' class to parse the time series into histogram and to call the WHAM
solver. Therefore, there should be different derived classes from the ``MDP''
class corresponding to different simulation software. Due to the limited amount
of time I have, I could only implement the derived class to handle the GROMACS
version 4.6.3 (or potentially any version with compatible output format) format
but I believe the extension is straightforward. The users are encouraged to
write their own data parser; as for users who are not proficient C++
programmer, it's also straightforward to transform your data into the supported
GROMACS format and user the GROMACS APIs directly (I'll give more details
in section~\ref{sec:usage}).
